%----------------------------------------------------------------------------------------
%	PACKAGES AND OTHER DOCUMENT CONFIGURATIONS
%----------------------------------------------------------------------------------------


\documentclass[a4paper]{article}
\usepackage[utf8]{inputenc} % Required for inputting international characters
\usepackage[T1]{fontenc} % Output font encoding for international characters
\usepackage[pdftex]{graphicx}     
\usepackage{mathpazo} % Palatino font
\usepackage[section]{placeins}
\usepackage[labelsep=quad,indention=10pt]{subfig}
\usepackage{url}
\usepackage{physics}
\newcommand\tab[1][1cm]{\hspace*{#1}}
\newcommand\sbullet[1][.5]{\mathbin{\vcenter{\hbox{\scalebox{#1}{$\bullet$}}}}}
\usepackage{amsmath}
\usepackage{color, colortbl}
\usepackage{tabularx,ragged2e,booktabs,caption}
\usepackage[table,x11names]{xcolor}
\usepackage[first=0,last=9]{lcg}
\newcommand{\ra}{\rand0.\arabic{rand}}
\captionsetup*[subfigure]{position=bottom}
\usepackage{tikz}
\usetikzlibrary{tikzmark}
\usepackage{float}







\begin{document}

%----------------------------------------------------------------------------------------
%	TITLE PAGE
%----------------------------------------------------------------------------------------

\begin{titlepage} % Suppresses displaying the page number on the title page and the subsequent page counts as page 1
	\newcommand{\HRule}{\rule{\linewidth}{0.5mm}} % Defines a new command for horizontal lines, change thickness here
	
	\center % Centre everything on the page
	
	%------------------------------------------------
	%	Headings
	%------------------------------------------------
	
	\textsc{\LARGE Fakultet elektrotehnike i računarstva}\\[1.5cm] % Main heading such as the name of your university/college
	
	\textsc{\Large Bioinformatika}\\[0.5cm] % Major heading such as course name
	
	
	
	%------------------------------------------------
	%	Title
	%------------------------------------------------
	
	\HRule\\[0.4cm]
	
	{\huge\bfseries Određivanje poravnanja parova sljedova korištenjem HMM}\\[0.4cm] % Title of your document
	
	\HRule\\[1.5cm]
	
	%------------------------------------------------
	%	Author(s)
	%------------------------------------------------
	
	\begin{minipage}{0.4\textwidth}
		\begin{flushleft}
			\large
			\textit{Autori}\\
			 {Tomislav Božurić}\\
			 {Martin Pisačić}\\
			 {Krešimir Topolovec} % Your name
		\end{flushleft}
	\end{minipage}
	~
	\begin{minipage}{0.4\textwidth}
		\begin{flushright}
			\large
			\textit{Zadatak}\\
			 doc. dr. sc. {Mirjana Domazet-Lošo} % Supervisor's name
		\end{flushright}
	\end{minipage}
	
	% If you don't want a supervisor, uncomment the two lines below and comment the code above
	%{\large\textit{Author}}\\
	%John \textsc{Smith} % Your name
	
	%------------------------------------------------
	%	Date
	%------------------------------------------------
	
	\vfill\vfill\vfill % Position the date 3/4 down the remaining page
	
	{\large Siječanj, 2019} % Date, change the \today to a set date if you want to be precise
	
	%------------------------------------------------
	%	Logo
	%------------------------------------------------
	
	\vfill\vfill
	\includegraphics[width=0.4\textwidth]{fer_logo.jpg}\\[1cm] % Include a department/university logo - this will require the graphicx package
	 
	%----------------------------------------------------------------------------------------
	
	\vfill % Push the date up 1/4 of the remaining page
	
\end{titlepage}






\section{Opis algoritma i vizualizacija}
U ovome radu korišten je modificirani Viterbijem algoritam pomoću kojeg korištnjem dinamičkog programiranja možemo pronaći najvjerojatniju sekvencu skrivenih stanja koja ujedno predstavlja optimalno poravnanje.
Da bismo klasičan Viterbijev algoritam tranformirali u HMM, moramo napraviti nekoliko izmjena. Prvo moramo odrediti vjerojatnosti za emitiranje simbola iz stanja. Npr. stanje M (match) ima vjerojatnosnu distribuciju $p_{ab}$ za emitaranje para simbola $ab$. Stanja $X$ i $Y$ imaju vjerojatnost emitiranja $p_a$ simbola $a$ umjesto praznine.
Također potrebno je definirati vjerojatnosti prijelaza između stanja tako da je suma vjerojatnosti odlaska iz pojedinog stanja jednaka 1. Vjerojatnost prijelaza iz stanja M u stanje X i Y opisujemo oznakom $\delta$, a vjerojatnost u ostanka u stanju X ili Y sa $\epsilon$.
Takva definicija ne definira kompletni model koji omogučava vjerojatnosnu distribuju po svim mogučim sekvecama. Za kompletiranje modela dodajemo početno i krajnje stanje $Početak$ i $kraj$. Definiramo vjerojatnost $\tau$ koja je jednaka za sve prijelaze iz stanaj M, X i Y u stanje End, te za stanje Begin u stanja M, X i Y.

\begin{figure}[H]
  \includegraphics[width=\linewidth]{HMM.png}
  \caption{Model HMM-a korištenog za poravnanje parova sekvenci}
  \label{fig:hmm}
\end{figure}


\textbf{Algoritam: Viterbijev algoritam za HMM} \cite{hmm_algorithms}\\
Inicijalizacija: \\
\tab \tab $v^M(0,0) = 1$\\
\tab \tab $v^{\sbullet[.75]}(i,0) = v^{\sbullet[.75]}(0,j) = 0$\\
Korak:\\
\tab za svaki $i=1,...,n$, $j=1,...,m$ \\
\begin{equation}
     v^M(i,j) = p_{x_iy_i} \max
    \begin{cases}
      (1 - 2\delta - \tau)v^M(i-1,j-1)\\
      (1-\epsilon-\tau)v^X(i-1,j-1)\\
      (1-\epsilon - \tau)v^Y(i-1,j-1)\\          
    \end{cases}
\end{equation}
\begin{equation}
     v^X(i,j) = q_{x_i}\max
    \begin{cases}
      \delta v^M(i-1,j)\\
      \epsilon v^X(i-1,j)\\
    \end{cases}   
\end{equation}
\begin{equation}
     v^Y(i,j) = q_{y_j}\max
    \begin{cases}
      \delta v^M(i,j-1)\\
      \epsilon v^Y(i,j-1)\\
    \end{cases}   
\end{equation}
Uvjet zaustavljanja:
$v^E = \max(v^M(n,m), v^X(n,m), v^Y(n,m))$ \\

Zbog činjenice da su vjerjoatnosti brojevi u intervalu $[0,1]$, gore navedeni algoritam nije upotrebiv za implementaciju na računalu zbog velikog broja množenja brojeva bliskih nuli, pa se u praksi koristi logaritamska inačica tog algoritma koja je navedena u nastavku.  

\textbf{Algoritam: optimalno poravnanje logaritamskih kvota} \cite{hmm_algorithms} 	 \\
Inicijalizacija: \\
\tab \tab $V^M(0,0) = - 2 log(\eta), V^X(0,0) = V^Y(0,0) = -\infty$\\
\tab \tab $V^{\sbullet[.75]}(i,-1) = V^{\sbullet[.75]}(-1,j) = -\infty$\\
Korak:\\
\tab za svaki $i=0,...,n$, $j=0,...,m$ osim (0,0):\\
\begin{equation}
     V^M(i,j) = s(x_i,y_j) + \max
    \begin{cases}
      V^M(i-1,j-1)\\
      V^X(i-1,j-1)\\
      V^Y(i-1,j-1)\\          
    \end{cases}
\end{equation}
\begin{equation}
     V^X(i,j) = \max
    \begin{cases}
      V^M(i-1,j) - d\\
      V^X(i-1,j) - e\\
    \end{cases}   
\end{equation}
\begin{equation}
     V^Y(i,j) = \max
    \begin{cases}
      V^M(i,j-1) -d\\
      V^Y(i,j-1) - e\\         
    \end{cases}
\end{equation}
Uvjet zaustavljanja:
$V = \max(V^M(n,m), V^X(n,m) + c , V^Y(n,m) + c )$ \\
Pri čemu su:\\
	\tab$s(a,b) = \log\frac{p_{ab}}{q_a q_b} + \log(\frac{1 - 2\delta - \tau }{(1-\eta)})$\\
    \tab$d = -\log\frac{\delta(1-\epsilon - \tau)}{(1-\eta)(1-2\delta - \tau)}$\\
    \tab$e = -\log\frac{\epsilon}{1-\eta}$\\
	\tab$c = \log( 1 - 2\delta - \tau) - \log(1-\epsilon - \tau)$
	
	
	
\begin{center}
\centering
\captionof{table}{Vizualizacija početnog stanja matrice $V_M$} \label{tab:title} 
  \begin{tabular}{ | >{\columncolor[gray]{0.8}}c | c | c | c | c | c | c |}
    \hline
     \rowcolor{lightgray} indeks & -1 & 0 & 1 & 2 & ... & m\\ \hline
     -1 & $-\infty$ & $-\infty$ & $-\infty$  & $-\infty$  & $-\infty$  & $-\infty$   \\ \hline
      0 & $-\infty$ & $-2log\eta$ & &  &  &\\ \hline
      1 & $-\infty$  & & & & &\\  \hline
      2 & $-\infty$  & & & & &\\ \hline
      ... & $-\infty$ & & & & &\\ \hline
      n & $-\infty$ & & & & &\\ \hline
      \end{tabular}
\end{center}

U tablici iznad možemo vidjet početno stanje matrice u kojoj pamtimo logaritamske kvote za stanje podudaranja. Iz modificiranog Viterbijevog algoritma koji koristiti logaritamske kvote iterativno korištenjem dinamičkog programiranja punimo tablicu i upisujemo novo izračunate vrijednosti. Analogno radimo za tablice ispod, te kada dođemo do kraja iste te tablice koristimo za rekonstrukciju optimalnog poravnanja. Želimo li izračunati primjerice za vrijednosti $i=1, j=0$ izraz za $V_M$ bi izgledao ovako: \\ 
\begin{equation}
     V^M(1,0) = s(x_1,y_0) + \max
    \begin{cases}
      V^M(0,-1) = -\infty\\
      V^X(0,-1) = -\infty\\
      V^Y(0,-1) = -\infty\\          
    \end{cases}
\end{equation}. Analogno popunjavamo preostali dio tablice. Ovdje možemo primjetiti kako za računanje vrijednosti u koraku $i,j$ nam trebaju podatci iz koraka $i-1, j-1$, tako da je potrebno pararelno popunjavati sve tri prikazane matrice.

\begin{center}
\centering
\captionof{table}{Vizualizacija početnog stanja matrice $V_X$} \label{tab:title} 
  \begin{tabular}{ | >{\columncolor[gray]{0.8}}c | c | c | c | c | c | c |}
    \hline
     \rowcolor{lightgray} indeks & -1 & 0 & 1 & 2 & ... & m\\ \hline
     -1 & $-\infty$ & $-\infty$ & $-\infty$  & $-\infty$  & $-\infty$  & $-\infty$   \\ \hline
      0 & $-\infty$ & $-\infty$ & &  &  &\\ \hline
      1 & $-\infty$  & & & & &\\  \hline
      2 & $-\infty$  & & & & &\\ \hline
      ... & $-\infty$ & & & & &\\ \hline
      n & $-\infty$ & & & & &\\ \hline
            
\hline
  \end{tabular}
\end{center}


\begin{center}
\centering
\captionof{table}{Vizualizacija početnog stanja matrice $V_Y$} \label{tab:title} 
  \begin{tabular}{ | >{\columncolor[gray]{0.8}}c | c | c | c | c | c | c |}
    \hline
     \rowcolor{lightgray} indeks & -1 & 0 & 1 & 2 & ... & m\\ \hline
     -1 & $-\infty$ & $-\infty$ & $-\infty$  & $-\infty$  & $-\infty$  & $-\infty$   \\ \hline
      0 & $-\infty$ & $-\infty$ & &  &  &\\ \hline
      1 & $-\infty$  & & & & &\\  \hline
      2 & $-\infty$  & & & & &\\ \hline
      ... & $-\infty$ & & & & &\\ \hline
      n & $-\infty$ & & & & &\\ \hline
            
\hline
  \end{tabular}
\end{center}

Kada smo završili s forward dijelom algoritma, odnosno kada su nam sve strukture podatka popunjene spremni za pronalazak optimalnog poravnanja dvije sekvence. Kako nam je uvjet "zaustavljanja" $V = \max(V^M(n,m), V^X(n,m) + c , V^Y(n,m) + c )$, tražimo maksimum između navednih stanja. Zatim   generiramo poravnanje  te u povratku prema početku ponavljamo postupak tražeći maksimum između $V_M, V_X, V_Y$.Na idućem primjeru je ilustriran primjer pronalaska optimalnog poravnanja gdje se na poziciji $(n,m)$ vrijednost računa kako je prethodno spomenuto. \\

\begin{center}
\centering
\captionof{table}{Vizualizacija pronalaska optimalnog poravnanja} \label{tab:title} 
  \begin{tabular}{ | >{\columncolor[gray]{0.8}}c | c | c | c |c  | c | c | c |}
    \hline
     \rowcolor{lightgray} indeks & 0 & 1 & 2 & 3 & ...&m-1  & m\\ \hline
      0 & $V^{\sbullet[.75]}(0,0)$ \tikzmark{d}   & &  &   &  & &   \\ \hline
      1 &  &$V^{\sbullet[.75]}(1,1)$ \tikzmark{c}  &  &   &  &  & \\ \hline
      2 &  & &  &  &  &   &\\ \hline
      3 &  & &  &   &     & &  \\ \hline
      ... &  & &  &   &  & &   \\ \hline
      n-1 &  & &  & ...  & & $V^{\sbullet[.75]}(n-1,m-1)$ \tikzmark{b}&   \\ \hline
      n &  & &  &   &  &  & $V^{\sbullet[.75]}(n,m)$ \tikzmark{a}\\ \hline
            
   
            
\hline
  \end{tabular}
  
  \begin{tikzpicture}[overlay, remember picture, yshift=.40\baselineskip, shorten >=.5pt, shorten <=.5pt]
     \draw [->] ({pic cs:a}) [bend right] to ({pic cs:b});
      \draw [->] ({pic cs:c}) [bend right] to ({pic cs:d});
  \end{tikzpicture}
\end{center}


\section{Analiza točnosti, vremena izvođenja i utroška memorije}

\section{Testiranje}
\subsection{Testiranje na sintetskim podatcima}
\subsection{Testiranje na stvarnim podatcima}

\newpage
\begin{thebibliography}{9}
\bibitem{hmm_applications} 
Byung-Jun Yoon. 
\textit{Hidden Markov Models and their Applications in Biological Sequence Analysis}.
\url{https://www.ncbi.nlm.nih.gov/pmc/articles/PMC2766791}, 
 US National Library of Medicine, National Institutes of Health, 2009.


\bibitem{hmm_algorithms} 
Jun Xie. 
\textit{Pairwise alignment using HMM}.
\url{http://www.stat.purdue.edu/~junxie/topic4.pdf},
 Purdue University.


\end{thebibliography}

\end{document}